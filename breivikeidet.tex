\documentclass[a4paper,12pt]{article}
\renewcommand{\baselinestretch}{1.2}
\DeclareMathSizes{12}{12}{12}{12} % {display size}{text size}{script size}{scriptscript size}.
% the above line is where you can declare mathe size in your text (whether you want it to be the same size of your font or bigger or smaller).
\usepackage{graphicx}	%	For graphics
\usepackage{amsmath}
%\usepackage{physics}	%	can be used for partial differentiation and others(needs to be installed)
\usepackage{nth}		%	For oridnal number in the text.
\usepackage{fmtcount}	%	For ordinal number in the text (I used this!)
\numberwithin{equation}{section} % This line adds the section numbers to eqs.

\begin{document}
\section{Introduction}

Planetary boundary layer (PBL) is located at the lowest part of the atmosphere, but that does not make it a boring subject to study. In fact, due to its location, it is home to variety of different physics mechanisms (e.g. turbulence, advection) and makes PBL one of the most exciting and challenging parts of the atmosphere to study. The majority of the weather patterns we observe every day are either the direct (e.g. inversion) or indirect (e.g. ) result of PBL. Therefore, a better understanding of PBL is of crucial importance in weather prediction studies. 

Height of the PBL is an important factor in defining the PBL. However, it has/is (been)/ found that PBL has no fixed and exact height and it can vary up to several kilometers in a matter of a few hours (in unstable atmosphere conditions) [source]. Generally, PBL is defined as the part of the atmosphere where the Earth's surface and atmosphere meet and interact (exchange physics) up to the height where these interactions are carried and felt by the atmosphere [Pleim2007b]. Along with height of the PBL, stability of the PBL is of special interest amongst meteorologists and climatologists. Stability status of the PBL can tell us about the buoyancy condition of the atmosphere and the possible related phenomena (e.g. inversion).

The aim of this work is to study the PBL of the Breivikeidet valley (location) and to find the relation between different physical parameters (mainly temperature, potential temperature, water vapor and wind velocity components with height) in the PBL over a course of 16 days (i.e. from March 10, 2016 to March 25, 2016). The investigated day were chosen in a way that a transition from warm weather to cold weather can be captured in order to study the behavior of the PBL at such weather transition. This work has made extensive use of the Weather Research and Forecasting (WRF) Model [source].

In chapter 2 a short theoretical background of the work is given. Chapter 3 focuses on the WRF settings. Chapter 4 and 5 present the result and discussion of the results respectively. Finally, Chapter 6 gives a short conclusion of the results.

\newpage

\section{Theoretical Background}

In the following section a short overview of the investigated parameters along with the important equations that were used to find the value of those parameters is given.

\subsection{Temperature}

Probably temperature is the most important parameter in PBL since any change in temperature can have immediate impact on the other parameters of the PBL. Therefore, a close attention was paid to temperature behavior in this study. In order to have a better view of the temperature evolution throughout the PBL, an attempt was made to investigate temperature behavior not only from the height-change perspective but also from radiation balance of the surface and radiative flux divergence views. 

\subsubsection{Radiation Balance}

Generally, the radiation in the PBL is PBL is highly effected by the radiation balance of the surface ground specially during daytime. Radiation balance in its simplest form considers the balance between the absorption and radiation of the ground at its surface as is stated by the eq.\ref{eq:1}.

\vspace{0.25cm}
\begin{equation}
R_N = R_{s\downarrow} + R_{s\uparrow} + R_{L\downarrow} + R_{L\uparrow}
\end{equation}
\vspace{0.25cm}

Where the $R_N$ is the net radiation at the surface, $R_{s\downarrow}$ and $R_{L\downarrow}$ are the incoming short and longwave radiation at the surface and $R_{s\uparrow}$ and $R_{L\uparrow}$ are the short and long wave radiation from the ground respectively. It has to be noted that the $R_{s\uparrow}$ is the partial reflection of the $R_{s\downarrow}$ by the surface. The net radiation absorbed and radiated, at all wavelengths, by the ground during a sufficiently long enough period will be zero (or very close to). 

Generally, during daytime shortwave from the sun is dominant and right after the sunset the longwave radiation of the ground will be dominant with their peak wavelength at $6000$ $K$ and $287$ $K$ for short and long wave radiation respectively (see Fig? and ? of the book Arya). During daytime the surface ground will warm up by absorbing the incoming shortwave radiation. At layers very close to the surface ($\sim 1 mm$) the generated heat by the ground will be transfered to the ambient air during what is called sensible heat transfer mechanism. Above that layer, the heat transfer process will be in a convective manner. This convection in turn will generate turbulence and results in vertical mixing of the PBL. As the sunset arrives, the ground will start to cool off by radiating in longwave radiation. And as a result, turbulence will slow down and the vertical mixing will weaken throughout the PBL.

\subsubsection{Radiative Flux Divergence}

Although radiation balance can give a simplified picture of what happens at the lowest part of the PBL close to the surface, it is not the complete picture. The concept of net radiative flux divergence/convergence is another tool that can help us to build a better picture of the influence of the radiation on different layers of air in the PBL at different heights. The rate of cooling or warming of a layer of air due to net radiation at PBL can be measured using the conservation of energy principle. Assuming a layer of air in PBL confined to a height of $z$ to $z$ $+$ $dz$, the cooling or warming rate of this layer of air due to net radiation effect can be found using the following equation: 

\vspace{0.25cm}
\begin{equation}
\Big(\frac{\partial T}{\partial t}\Big)_R = \Big(\frac{1}{\rho c_p}\Big)\Big(\frac{\partial R_N}{\partial z}\Big)
\end{equation}
\vspace{0.25cm}

Where $\Big(\frac{\partial T}{\partial t}\Big)_R$ is the rate of cooling/warming of an specific layer of air in PBL, $\Big(\frac{\partial R_N}{\partial z}\Big)$ $<$ 0 is the convergence rate of net radiation at height $z$ and results in air warming off and $\Big(\frac{\partial R_N}{\partial z}\Big)$ $>$ 0 is the divergence rate of net radiation at height $z$ and results in air cooling off (see fig).

Several studies have found that the contribution of the shortwave radiation to the net radiative flux divergence/convergence is quite insignificant and vertical shortwave radiative heating is nearly equally distributed in the surface boundary layer (phd and references therein). On the other hand, the longwave radiative flux is highly height dependent and strongly influences the air layers in the PBL. This study will focus on the effect of the net longwave radiation on PBL at different heights.

\subsubsection{Mixing Ratio, Saturated Mixing Ratio, Relative Humidity, and Dew Point Temperature}

Mixing ratio, $w$, is one of the wrf's outputs. It is defined as the mass of the water vapor (in gram) to the mass of the dry air (in kilogram). However, mixing ratio alone may not provide a clear picture on humidity of the PBL. To have a clearer view, one can use relative humidity (RH) as a tracer of mixing ratio, saturation mixing ratio, and dew point temperature, $T_{dp}$, in the PBL. Also, it can be used to trace precipitation in the atmosphere. As an air parcel rises in the atmosphere the water vapor content of the parcel approaches its dew point. (the temperature below which water vapor goes through a phase transition and condenses into water.) However, before that it the water vapor in the parcel reaches its saturation limit. It is defined as the limit where the water evaporation rate is equal to the water condensation rate. At this point the water vapor pressure of the parcel can be estimated using eq.\ref{eq:?}[paper refe]. 

\vspace{0.25cm}
\begin{equation}\label{eq:3}
e_s = [1.0007 + (3.46 \times 10^{-6} P)] \times 6.1121 \: exp\Big(\frac{17.502\; T}{240.97 + T}\Big)
\end{equation}
\vspace{0.25cm}

Where $e_s$ is water vapor pressure in $hPa$, $P$ is pressure in $hPa$, and $T$ temperature in degree Celsius $^\circ C$.
Once water vapor pressure is know the saturation mixing ratio at each temperature and height can be found from the following equation:

\vspace{0.25cm}
\begin{equation}\label{eq:4}
w_s = 0.622 \: \frac{e_s}{P - e_s}
\end{equation}
\vspace{0.25cm}

Where $w_s$ is the saturation mixing ratio, $P$ is pressure at each height, and $e_s$ is the water vapor ratio from eq.\ref{eq:3}. Once we have mixing ratio and saturation mixing ratio, relative humidity (RH) can be easily found from the following equation:

\vspace{0.25cm}
\begin{equation}
RH\equiv\frac{w}{w_s}
\end{equation}
\vspace{0.25cm}

Where $w$ and $w_s$ are mixing ratio and saturated mixing ratio respectively. Finally, the dew point temperature can be found using Arden Buck equation (refe).

\vspace{0.25cm}
\begin{equation}
T_{dp} = \frac{c\gamma_m(T,RH)}{b - \gamma_m (T,RH)}
\end{equation}

where,

\begin{equation}
\gamma_m(T,RH) = ln\Bigg(\frac{RH}{100}\: exp\Big(\Big(b - \frac{T}{d}\Big)\Big(\frac{T}{c+T}\Big)\Big)\Bigg)
\end{equation}

\vspace{0.25cm}

with the following set of constants,
\begin{equation*}
(a = 6.1121\: \text{millibar;} \quad b = 18.678; \quad c = 257.14 \:^\circ C; \quad d = 234.5 \: ^\circ C)
\end{equation*}

\vspace{0.25cm}

With the above set of constants the results of the eq 2.6 will have a maximum error of $1\: \%$. However, in has to be mentioned that there are other sets of constants with lesser degree of errors but limited to a certain range of temperatures. A comparison between temperature, dew point, and humidity can give us clue as to what parts of PBL has gone through precipitation.


\subsection{Potential Temperature}

Potential temperature ($\Theta$) is probably the most interesting parameter in this study. Potential temperature is defined as the temperature that an air parcel would have if it is brought back to a reference pressure level adiabatically.

\vspace{0.5cm}

\begin{equation}\label{eq:1}
\Theta = T(\frac{P_R}{P})^K
\end{equation}

\vspace{0.5cm}

In equation \ref{eq:1} $P_R$ is the reference pressure level and usually replaced by $1000$ (pressure at sea level and in mbar). P and T are pressure and temperature of the ambient environment at some point in the atmosphere. Also, $K$ is the Kappa exponent and, for dry air, it is $K = \frac{R}{C_P}\cong 0.286$.

The importance of potential temperature comes from the fact that it is not affected by the physical ascending or descending of an air parcel (unlike actual temperature). Potential temperature is a measure of stratification of the atmosphere which in turn it shows how stable the atmosphere is. In other words, potential temperature is a measure of the resistivity of the atmosphere to the vertical motion of an air parcel. Depending how resistant the atmosphere is to the vertical motion of an air parcel,potential temperature can be positive ($\Theta > 0$), neutral ($\Theta = 0$) or negative ($\Theta < 0$). If an air parcel is released and given an initial push (toward higher altitudes) in an atmosphere with positive potential temperature, the vertical motion of the air parcel will be suppressed and it will be pushed back to its initial location. Such resistant atmosphere is called an stably stratified atmosphere. In such atmosphere the actual temperature of the ambient is higher than that of the air parcel. As a result of this temperature difference and following the Buoyancy acceleration (eq. \ref{eq:2}) a downward force will act upon the air parcel.

\vspace{0.5cm}
\begin{equation}\label{eq:2}
a_b = g\bigg(\frac{T_p - T_a}{T_a}\bigg)
\end{equation}

\vspace{0.25cm}

Where $a_b$ is the buoyancy acceleration, $g$ is acceleration due to gravity and $T_p$ and $T_a$ are the actual temperature of the air parcel and atmosphere respectively. In other words if potential temperature increases with height, it is a sign that the air parcel is entering a stable atmosphere. On the other hand, if the air parcel on its way to higher altitudes experiences a temperature difference in a way that the temperature inside the parcel is higher than the ambient temperature of the atmosphere, the buoyancy force will be upward and air parcel will keep rising toward higher altitudes. An atmosphere in which an air parcel can travel toward higher altitudes is called an unstably stratified atmosphere. If an air parcel is located in some part of the atmosphere where there is no temperature difference between the air parcel and the ambient atmosphere then there will be no buoyancy force acting upon the air parcel and the air parcel will not move. Such atmosphere is called a neutral atmosphere.

\newpage

\section{Weather Research and Forecast (WRF) model}

Weather Research and Forecast (WRF)\footnote{http://www.wrf-model.org} is a numerical weather model that is mainly used for weather research and forecast purposes. 
For the purpose this work WRF model version 3.8 (the latest version at the time of writing this report) was used. Below, different parameters of the WRF model that were used in this work are explained shortly, but the reader is encouraged to see [tor] and [WRF user] for a detailed discussion of the choice of different parameters used in WRF model. Also, appendix ... shows the choice of different parameters for the purpose of this work in the form of a namelist.input file. 

%\subsection{General Structure of WRF}
%WRF is a Gudnov based model. Explain i,j and k also the middle of the cell %parameters.

\subsection{WRF settings}
Generally, settings in the WRF model can be divided into ... categories where we shortly discuss some of them in the following.

\subsubsection{Domains and Dimensions}
For the purpose of this study a three-nested-domain model was used with the following grid points length; domain one (i.e. the outermost domain) $dx = 18000$ $m$, domain two (i.e. middle domain) $dx = 6000$ $m$, and third domain (i.e. the inner most domain with finest resolution) $dx = 2000$ $m$. The model covers an area from long lat ... with nn and mm grid points in x and y direction respectively. Also, at each grid point, fifty cells are stacked on top of one another. This way WRF model will calculate some of the parameters (e.g. potential temperature, humidity) at fifty different heights.

\subsubsection{Time Step}
WRF model is based on a 3rd order Runge-Kutta integration [53]. Therefore, time step is one of the core parameters of this model and special attention has to be paid to selecting the right time step. A very small time step can be computationally very expensive and a very big time step can cause an early divergence which in turn causes the simulation to crash. In order to find the appropriate value for the time step the Courant-Fredrichs-Lewy (CFL) condition and maximum Courant number by  was followed [21]. They introduced a minimum and maximum value for the time step (eq. 3.1) based on which the value of $dt$ should be confined to a certain interval.
\vspace{0.25cm}
\begin{equation}
t = ...
\end{equation}

\vspace{0.25cm}

Finally, after several test runs the value of $dt$ $=$ $60$ seconds was chosen as the value of the time step.

\subsubsection{Planetary Boundary Layer Model}

For the 

\newpage

\section{Discussion}

\subsection{Choice of Days}
This chapter will focus on stability and turbulence state of the PBL at Breivikeidet. In order to keep this report short only three days of the sixteen days were analysed but a complete list of figures of all days can be found in appendix ?. The three days are March \ordinalnum{15}, \ordinalnum{16}, and \ordinalnum{23}. The reason behind this choice of days is to study the behavior of PBL of the warmest, coldest and a transition day (between the warmest day and a cold day). Figure ?, shows a plot of the maximum, minimum and the average temperature of all days.

\begin{figure}[H]
	\includegraphics[width = \textwidth, height = 5cm, trim={0 0 0 5cm}]{/home/ubuntu/nozhan/science/tromuni/semester/s17/individual/report/plot/temp/temp_av.png}
\end{figure}

The main focus of this chapter will be on the evolution of the four physical parameters with height and time (i.e. three-hour time intervals) in the atmosphere up to the height of $\sim$ 900 $m$. These parameters are water vapor mixing ratio, potential temperature, temperature, and wind speed.

A suspicious feature is observable at midnight time (00:00) at the water mixing ratio, potential temperature and temperature evolution graphs. A further analysis of the feature is discussed at the end of this chapter (see ) as to whether this feature is a computational artifact or it truly is the behavior of the PBL at this time of the day. However, to avoid the risk of miss-interpretation of the data, the behavior of the parameters at this time of the day will not be discussed.

\subsection{The Warmest Day (March 15, 2016)}

This day is the warmest day among all the sixteens day and is preceded by another warm day and followed by a cold (the transition day; see fFig.?).

\subsubsection{Temperature}

Figure ?? shows distribution of the temperature at different hours of the day at different heights. It is clear that as we go through the day temperature decreases both with time and height except the slight inclination of the temperature towards higher temperatures at 06:00 at $\sim$ 600. As discussed in section ... this could be the result of latent heat release by the clouds at that altitude. From the surface to 100 $\sim$ $m$ the temperature profile appears to be constant at different hours of the day. The possible explanation for this behavior could be that as the sun insolation heats the ground in the morning it starts heating up the surface. The generated heat from the surface causes vertical convection which in turn results in vertical mixing of the air layers in PBL up to $\sim$ 100 $m$. However, looking at fig.? it is clear that the surface temperature does not vary much (mostly constant) during the day. Also, the 2 $m$ temperature profile shows an almost constant decrease with time during the day. It suggests that there is not much convection going on above the ground. This makes it hard for the heat convection mechanism to be the possible reason behind the observed constant temperature in the PBL. Another possible explanation could be mechanically generated turbulent as a result of wind friction by the surface and the resulting wind shear through the PBL. Although, looking at the fig.?? and different wind components there seems to be no wind shear up to $\sim$ 100 $m$. It could be that a combination of heat convection and mechanical mixing (or a third unnoticed mechanism) is the possible reason behind the constant temperature profile.

It might be of importance to see how the energy budget of the surface changes during the day. Fig? shows the energy budget profile of the surface skin at March 15. There are several features that immediately attract attention. The bump in the middle of the day, the incoming larger-than-shortwave longwave in the middle of the day, and several troughs in the longwave radiation at the end of the day.

In a day with clear skies the incoming shortwave is by far stronger than the incoming longwave radiation. However, in our case the incoming longwave radiation is stronger. This could mean that March 15 could have been a cloudy day. The clouds have absorbed part of the incoming shortwave radiation and re-emitted in logwave radiation. Also, some part of the incoming shortwave has been reflected by the clouds. These two factors cause the strong incoming longwave and the weaker shortwave radiation. As, previously, mentioned there are several periods at which the incoming longwave radiation undergoes a sudden decrease along with a slight increase in the incoming shortwave radiation (e.g. 08:00, 16:00) with the longwave radiation still the dominant incoming radiation. It is possible that at these hours the sky has cleared (or at least clouds have thinned out) for several minuets. With less/no cloud in the way there will be less absorption (and re-emission in infrared) and less reflection of the shortwave radiation. As a result, an increase and decrease in the incoming short and longwave radiation occurs respectively. Also, it has to be noted that since the shortwave radiation is not the dominant radiation at these time intervals, it may well be that the sunlight does not hit the surface directly (supporting the cloud thinning scenario).

On the outgoing radiation side, the outgoing shortwave radiation is a fraction of the incoming shortwave radiation\footnote[56]{The WRF simulation does not output the outgoing shortwave radiation but it does output the albedo. The $S_{\uparrow} = (1 - a)\: S_{\downarrow}$ was used to find the outgoing shortwave radiation where $a$ is the albedo.} depending on the value of albedo of the surface at each hour. Both incoming and outgoing shortwave radiations drop to zero after sunset. At night the surface cooling starts. The surface tries to get rid of the energy that has absorbed during the day by radiating it away at infrared\footnote{Longwave radiation ($L_{\uparrow}$), also, is not provided by the WRF. The surface skin temperature was used to find the $L_{\uparrow}$ using Stefan-Boltzmann equation ($R = \sigma \: T^4$).}. 

Finally, the black line shows the net energy budget of the surface at March 15.As expected, in the middle of the day the energy budget of the surface raises towards positive values a sign that the surface is receiving more energy than it is emitting and the surface temperature rises. As the sunset approaches the energy budget of the surface moves towards negative values a meaning that the surface is loosing more energy than it receives a sign that the surface is cooling.


One can calculate the latent heat released in the atm. and the equivalent increase in the long wave regime and the effect of the on the ground but it is out of the scope of this work.

\subsubsection{Mixing Ratio, Saturated Mixing Ratio, Relative Humidity, and Dew point}

The initial analysis of the mixing ratio $q$ is similar to temperature. Mixing ratio decreases with time at all altitudes. However, at 06:00 there is a shift in mixing ratio towards lower values around 700 $m$ height. Looking at temperature and potential temperature plots around the same height a sudden increase is noticeable. In order to investigate this feature (and probably other hidden features) more the procedure in 2.1.3 was followed to find the relative humidity ($RH$) and dew point temperature ($T_{dp}$). Figure ?? shows a graph of potential temperature, relative humidity, temperature and dew point. The strong and sudden change of $RH$ towards lower values is observable around $700$ $m$. Also, convergence of $RH$ at different hours of the day, except 21:00, toward 100 $\%$ at around ... $m$ is another noticeable feature the can be interpreted as the lifting condensation level (LCL)\footnote{Lifting condensation level (LCL) is the height up to which an air parcel's temperature will decrease under adiabatic lapse rate and once it reaches this level the air parcel will have $RH$ of 100 $\%$.} of this day. Considering the change in $RH$, potential temperature, temperature, and dew point at 06:00 it is highly probable that a precipitation has occurred at this hour. The air in PBL at this hour has reached the LCL and moved above this level. Once it has moved above this level the temperature of the air has dropped below its dew point and as a result the water vapor content of the air has released energy and condensed into water and dropped out of the air in form of precipitation. The released energy in form of latent heat from the air has resulted in an increases in the temperature of the atmosphere and, consequently, potential temperature (see. fig ...). To test this scenario the weather stations website was checked. It appears that there has been a period of precipitation in the amounts of 0.2 $mm$ and 0.1 $mm$ at 06:00 and 07:00 respectively in the morning of the March 15.
 

%https://en.wikipedia.org/wiki/Arden_Buck_equation
\end{document}